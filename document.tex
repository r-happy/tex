\documentclass[paper=a4paper]{jlreq}

%画像
\usepackage{graphicx}
%ハイパーリンク
\usepackage[hidelinks]{hyperref}
%表
\usepackage{makecell}
%数式
\usepackage{amsmath}
\usepackage{mathtools}

\usepackage[]{luatexja-preset}
\usepackage{float}
\usepackage{listings, jvlisting}
\usepackage{multirow}
\usepackage{booktabs}
\usepackage{amssymb}

% --- command ---
\newcommand{\TITLE}{title is here}
\newcommand{\NAME}{name is here}
\newcommand{\NUMBER}{class number is here}
\newcommand{\WEATHER}{weather is here}
\newcommand{\TEMPERATURE}{temperature is here}
\newcommand{\HUMIDITY}{humidity is here}
\newcommand{\Figure}[4]{
  \begin{figure}[H]
    \centering
    \includegraphics[width=#1\linewidth]{public/img/#2}
    \caption{#3}
    \label{fig:#4}
  \end{figure}
}
\newcommand{\Inline}[1]{\lstinline[style=inline]{#1}}
\renewcommand{\lstlistingname}{ソースコード}

\newenvironment{Summary}{\section{要旨}}{}

% --- config ---
% \setmonofont{PlemolJP35 Console NF}
% \setmonojfont{PlemolJP35 Console NF}
\lstset{
  % タブの展開後のサイズ
  tabsize={4},
  % 行番号表示,デフォルト: none 他のオプション: left, right
  numbers=left,
  % 識別子の書体指定
  identifierstyle={\small},
  % 行番号の書体指定
  %numberstyle=\scriptsize,
  % 注釈の書体。
  commentstyle={\small\ttfamily},
  ndkeywordstyle={\small},
  columns=[l]{fullflexible},
  xrightmargin=0\zw,
  xleftmargin=0\zw,
  numbersep=1\zw,
  %backgroundcolor={\color[gray]{.85}},
  % frameの指定.デフォルト: none 他のオプション: leftline, topline, bottomline, lines, single, shadowbox
  frame=lines,
  % 行が長くなってしまった場合の改行.デフォルト: false
  breaklines=true,
  %標準の書体
 	basicstyle = \ttfamily\scriptsize\linespread{1.0}\normalsize\small,
  keepspaces=true
}
\lstdefinestyle{inline}{
  basicstyle=\ttfamily\small,
  breaklines=true,
  columns=fullflexible,
}

% --- begin ---
\begin{document}

% {\centering \LARGE \TITLE \par}

% \begin{table}[b]
%     \centering
%     \begin{tabular}{|c|c|}
%         \hline
%         報告者   & \NAME \quad \NUMBER \\
%         \hline
%         共同実験者 & \NAME \quad \NUMBER \\
%               & \NAME \quad \NUMBER \\
%               & \NAME \quad \NUMBER \\
%               & \NAME \quad \NUMBER \\
%               & \NAME \quad \NUMBER \\
%         \hline
%         天候    & \WEATHER            \\
%         \hline
%         室温    & \TEMPERATURE        \\
%         \hline
%         湿度    & \HUMIDITY           \\
%         \hline
%         提出日   & \today              \\
%         \hline
%     \end{tabular}
% \end{table}

% \newpage

% {\centering \Large \TITLE \par}
% \rightline{提出者 : \NUMBER \quad \NAME}
% \rightline{提出日 : \today}

\begin{Summary}
  本実験ではレーザー光の干渉と回折を理解し、レーザー光の波長と単スリットのスリット幅を測定した。
  レーザー光の波長を計算した結果は$6.31 \times 10^{-5} \, [\mathrm{cm}]$であった。
  レーザー光の波長の理論値は$632.8 \, [\mathrm{nm}](6.32 \times \, 10^{-5} \, [\mathrm{cm}])$であるため、実験値は理論値とほぼ一致している。
  また、単スリットのスリット幅を計算した結果は$1.8 \times 10^{-2} \, \mathrm{cm}$であった。
  単スリットのスリット幅を遊動顕微鏡で測定した結果は$2.0 \times 10^{-2} \, \mathrm{cm}$であるため、実験値は測定値とほぼ一致している。
\end{Summary}

\section{目的}
レーザー光の干渉と回折を理解し、干渉縞の間隔などからレーザー光の波長、単スリットのスリット幅を測定する。

\section{実験手順}
本実験の実験手順は追加資料\textnormal{\textup{pp.}\,\textnormal{9--12}}を参照した。
ただし、ランプスケールで干渉縞を測定する際には干渉縞の両端を測定する。
これは干渉縞の中心というのは人によって異なるため、干渉縞の両端を測定することで正確に干渉縞の幅を測定するためである。
例として$X_{0}$から$X_{+1}$までの距離$D_{+1}$、$X_{0}$から$X_{-1}$までの距離$D_{-1}$を測定する場合を図\ref{fig:measure}に示す。
\textcircled{\scriptsize 5}は左端から左端までの距離、\textcircled{\scriptsize 6}は右端から右端までの距離である。
求めたい次数1の$D_{1}$は\textcircled{\scriptsize 5}と\textcircled{\scriptsize 6}の和の平均値である。
よって求めたい次数mの$D_{m}$は式\eqref{eq:dm}のように表される。
\begin{equation}
  \label{eq:dm}
  \begin{aligned}
    D_{m} & = \frac{\textcircled{\scriptsize 5} + \textcircled{\scriptsize 6}}{4}
  \end{aligned}
\end{equation}
\Figure{0.8}{ruler.drawio.png}{干渉縞の測定}{measure}
またグラフ用紙に干渉縞を記録する際にも干渉縞の両端を記録することで正確に干渉縞の幅を測定することができる。
計算方法は式\eqref{eq:dm}と同様である。

\end{document}