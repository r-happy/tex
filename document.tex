\documentclass[paper=a4paper]{jlreq}

%画像
\usepackage{graphicx}
%ハイパーリンク
\usepackage[hidelinks]{hyperref}
%表
\usepackage{makecell}
%数式
\usepackage{amsmath}
\usepackage{mathtools}

\usepackage[]{luatexja-preset}
\usepackage{float}
\usepackage{listings, jvlisting}
\usepackage{multirow}
\usepackage{booktabs}
\usepackage{amssymb}

% --- command ---
\newcommand{\TITLE}{title is here}
\newcommand{\NAME}{name is here}
\newcommand{\NUMBER}{class number is here}
\newcommand{\WEATHER}{weather is here}
\newcommand{\TEMPERATURE}{temperature is here}
\newcommand{\HUMIDITY}{humidity is here}
\newcommand{\Figure}[4]{
  \begin{figure}[H]
    \centering
    \includegraphics[width=#1\linewidth]{public/img/#2}
    \caption{#3}
    \label{fig:#4}
  \end{figure}
}
\newcommand{\Inline}[1]{\lstinline[style=inline]{#1}}
\renewcommand{\lstlistingname}{ソースコード}

\newenvironment{Summary}{\section{要旨}}{}

% --- config ---
% \setmonofont{PlemolJP35 Console NF}
% \setmonojfont{PlemolJP35 Console NF}
\lstset{
  % タブの展開後のサイズ
  tabsize={4},
  % 行番号表示,デフォルト: none 他のオプション: left, right
  numbers=left,
  % 識別子の書体指定
  identifierstyle={\small},
  % 行番号の書体指定
  %numberstyle=\scriptsize,
  % 注釈の書体。
  commentstyle={\small\ttfamily},
  ndkeywordstyle={\small},
  columns=[l]{fullflexible},
  xrightmargin=0\zw,
  xleftmargin=0\zw,
  numbersep=1\zw,
  %backgroundcolor={\color[gray]{.85}},
  % frameの指定.デフォルト: none 他のオプション: leftline, topline, bottomline, lines, single, shadowbox
  frame=lines,
  % 行が長くなってしまった場合の改行.デフォルト: false
  breaklines=true,
  %標準の書体
 	basicstyle = \ttfamily\scriptsize\linespread{1.0}\normalsize\small,
  keepspaces=true
}
\lstdefinestyle{inline}{
  basicstyle=\ttfamily\small,
  breaklines=true,
  columns=fullflexible,
}

% --- begin ---
\begin{document}

{\centering \LARGE \TITLE \par}

\begin{table}[b]
  \centering
  \begin{tabular}{|c|c|}
    \hline
    報告者   & \NAME \quad \NUMBER \\
    \hline
    共同実験者 & \NAME \quad \NUMBER \\
          & \NAME \quad \NUMBER \\
          & \NAME \quad \NUMBER \\
          & \NAME \quad \NUMBER \\
          & \NAME \quad \NUMBER \\
    \hline
    天候    & \WEATHER            \\
    \hline
    室温    & \TEMPERATURE        \\
    \hline
    湿度    & \HUMIDITY           \\
    \hline
    提出日   & \today              \\
    \hline
  \end{tabular}
\end{table}

\newpage

% {\centering \Large \TITLE \par}
% \rightline{提出者 : \NUMBER \quad \NAME}
% \rightline{提出日 : \today}

\section{画像}
図\ref{fig:rhappy_icon}にらっぴーのアイコン示す。
\Figure{0.5}{rhappy_icon.jpg}{ラッピーのアイコン}{rhappy_icon}

\section{ソースコード}
\subsection{ファイルの一部分}
\Inline{print__hello_world}関数のソースコードをリスト\ref{lst:print__hello_world_func}に示す。

\begin{lstlisting}[caption={print\_\_hello\_world関数のソースコード}, label={lst:print__hello_world_func}]
void print__hello_world(void) {
    // hello worldを出力
    printf("hello world\n");
}
\end{lstlisting}

\subsection{ファイルを読み込む}
\lstinputlisting[caption={main.cのソースコード}, label={lst:source_main}]{public/code/main.c}

\subsection{文章内でmonoフォントを使う}
sample text sample text \Inline{sample text} sample text


\end{document}
